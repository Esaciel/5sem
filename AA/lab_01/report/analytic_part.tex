\chapter{Аналитическая часть}

В данном разделе будут приведены понятия матриц, их произведения, классический алгоритм умножения матриц, алгоритм Винограда и его оптимизированная версия.

\section{Матрица}

\textbf{Матрица} --- математический объект, записываемый в виде прямоугольной таблицы элементов кольца или поля (например, целых или комплексных чисел), которая представляет собой совокупность строк и столбцов, на пересечении которых находятся её элементы. Количество строк и столбцов матрицы задают размер матрицы.~\cite{book:matrix}

\begin{equation}
	A_{m \times n} =
	\begin{pmatrix}
		a_{11} & a_{12} & \ldots & a_{1n}\\
		a_{21} & a_{22} & \ldots & a_{2n}\\
		\vdots & \vdots & \ddots & \vdots\\
		a_{m1} & a_{m2} & \ldots & a_{mn}
	\end{pmatrix}
	\label{equ:matrix_definition}
\end{equation}

$a_{ij}$ --- элемент матрицы, находящийся в \textit{i-ой} строке и \textit{j-ом} столбце.


Для матриц определены следующие математические операции:
\begin{enumerate}[label={\arabic*)}]
	\item сложение с матрицей идентичного размера;
	\item умножение на скаляр;
	\item произведение двух матриц.
\end{enumerate}

\emph{Произведением матриц} $A_{n \times q}$ и $B_{q \times m}$ называется матрица $C_{n \times m}$ (обозначается $C = A \cdot B$), каждый элемент $c_{ij}$ которой равен сумме произведений элементов $i$-й строки матрицы $A$ на соответствующие элементы $j$-о столбца матрицы $B$:

\begin{equation}
	\forall i = \overline{1 \ldots n},\ \forall j = \overline{1 \ldots m} \quad c_{ij} = \sum_{k=1}^{p} a_{ik} \cdot b_{kj}
	\label{equ:matrix_product}
\end{equation}

При этом операция умножения $A_{n \times q} \cdot B_{k \times m}$ не определена в случае $q \neq k$

\textit{Замечание:} операция умножения $A_{n \times m} \cdot B_{m \times n}$ в общем случае не коммутативна, то есть $A \cdot B \ne B \cdot A$

\section{Классический алгоритм умножения матриц}

Классический алгоритм умножения матриц вытекает из математического определения и реализует формулу~\eqref{equ:matrix_product}. Асимптотическая сложность такого алгоритма равна $O(n^3)$ для двух матриц порядка $n \times n$~\cite{book:stdmul_complexity}.

\section{Алгоритм Винограда умножения матриц}

Каждый элемент результирующей матрицы $c_{ij}$ представляет собой скалярное произведение $i$-ой строки матрицы $A$ на $j$-й столбец матрицы $B$. Подобное умножение допускает предварительную подготовку данных для уменьшения суммарного числа операций умножения~\cite{book:Winograd}

Для двух векторов  $V$ и $W$ длины $k$ каждый, скалярное произведение определяется как: $s = V \cdot W = \sum_{i=1}^{k} v_{i} \cdot w_{i}$.

Тот же результат можно получить следующей формулой:

\begin{equation}
	\label{equ:Winograd}
	s = \sum_{i=1}^{\lfloor k/2 \rfloor}{(v_{2i} + w_{2i + 1})\cdot(v_{2i+1} + w_{2i})} - \sum_{i=1}^{\lfloor k/2 \rfloor}{v_{i}*v_{2i}} - \sum_{i=1}^{\lfloor k/2 \rfloor}{w_{i}*w_{2i}} + (w_{k}\cdot v_{k} \cdot k \bmod 2)
\end{equation}

Несмотря на то, что суммарное число операций в формуле~\eqref{equ:Winograd} больше, 2-е и 3-е слагаемые допускают предварительную обработку, так как зависят одновременно только от рядов или колонн одной матрицы. Так можно уменьшить итоговое число умножений, и, поскольку для ЭВМ (электронная вычислительная машина) операция умножения намного более ресурсоёмкая, чем операция сложения, реализация такого алгоритма на практике должна быть быстрее стандартной.

\section{Оптимизация алгоритма Винограда}

При программной реализации алгоритма Винограда, согласно варианту требуется провести следующие оптимизации:
\begin{enumerate}[label=\arabic*)]
	\item избавиться от массива MulH, вычисляя значение для каждой отдельный строки (то есть избавится от предварительного вычисления 2-ого слагаемого в формуле~\eqref{equ:Winograd}, вычисляя его одновременно с первым);
	\item заменить умножения $x \cdot 2$ на побитовый сдвиг $x << 1$.
\end{enumerate}

\section*{Вывод}

В данном разделе были приведены понятия матриц и операция их умножения, стандартный алгоритм умножения матриц, алгоритм Винограда для умножения матриц и его оптимизация.

\clearpage
