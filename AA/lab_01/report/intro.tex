\ssr{ВВЕДЕНИЕ}

Операция умножения матриц является одной из фундаментальных и наиболее ресурсоёмких задач в линейной алгебре и вычислительной математике. Она находит применение в различных областях науки и техники, включая решение систем линейных уравнений, компьютерную графику, машинное обучение, обработку сигналов и математическое моделирование систем.

Целью данной лабораторной работы является исследование алгоритмов умножения матриц.

Для достижения поставленной цели необходимо решить следующие задачи:
\begin{enumerate}[label={\arabic*)}]
	\item описать три алгоритма умножения матриц;
	\item создать программное обеспечение, реализующее следующие алгоритмы:
	\begin{itemize}[label=---]
		\item классический алгоритм умножения матриц;
		\item алгоритм Винограда;
		\item оптимизированный алгоритм Винограда;
	\end{itemize}
	\item оценить трудоёмкость полученных реализаций;
	\item провести анализ временных затрат работы программы и выявить влияющие на них факторы;
	\item провести сравнительный анализ алгоритмов.
\end{enumerate}

\clearpage
