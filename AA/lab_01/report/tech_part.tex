\chapter{Технологическая часть}

В данном разделе будут перечислены средства реализации, листинги кода
и функциональные тесты.

\section{Средства реализации}

В качестве языка программирования для выполнения данной лабораторной работы был выбран С++~\cite{book:cpp}, так как он является статически типизированным языком с гарантированным временем выполнения, что соотносится с требованиями к лабораторной работе.

Для замера процессорного времени выполнения программы используется функция \textit{clock()} библиотеки \textit{ctime}~\cite{book:ctime_sync}~\cite{book:time_clock}

\section{Сведения о модулях программы}

Программа состоит из 6 модулей:

\begin{enumerate}[label={\arabic*)}]
	\item definitions.hpp --- модуль, содержащий определения типов данных;
	\item err.hpp --- модуль, содержащий описание классов исключений;
	\item extras.cpp --- модуль, содержащий функции инициализации объектов;
	\item solution.cpp --- модуль, содержащий реализации всех алгоритмов;
	\item io.cpp --- модуль, содержащий функции ввода-вывода;
	\item main.cpp --- точка входа программы;
\end{enumerate}

\section{Реализация алгоритмов}

В листинге~\ref{lst:defs} содержится описание используемых типов данных, в листингах~\ref{lst:std}~--~\ref{lst:WinOpt} приведены реализации алгоритмов умножения матриц

\lstinputlisting[label=lst:defs,caption=Используемые типы данных]{listings/defs.txt}
\clearpage

\lstinputlisting[label=lst:std,caption=Реализация стандартного алгоритма умножения матриц]{listings/stdMult.txt}
\clearpage

\lstinputlisting[label=lst:Win,caption=Реализация алгоритма Винограда умножения матриц]{listings/Win.txt}
\clearpage

\lstinputlisting[label=lst:WinOpt,caption=Реализация оптимизированного алгоритма Винограда умножения матриц]{listings/WinOpt.txt}
\clearpage

\section{Функциональные тесты}

Тестирование проведено по методологии чёрного ящика. 

В ходе выполнения программы тестовые данные подаются на вход всем трём реализациям; совпадение результатов их работы проверяется внутри программы. Функциональные тесты приведены в таблице~\ref{tbl:func_tests}

\begin{table}[ht]
	\small
	\begin{center}
		\begin{threeparttable}
			\caption{Функциональные тесты для реализаций алгоритмов умножения матриц}
			\label{tbl:func_tests}
			\begin{tabular}{|c|c|c|c|c|}
				\hline
				\multicolumn{2}{|c|}{\bfseries Входные данные}
				& \multicolumn{2}{c|}{\bfseries Результат для классического алгоритма} \\
				\hline 
				\bfseries Матрица 1
				& \bfseries Матрица 2
				& \bfseries Ожидаемый результат
				& \bfseries Фактический результат \\
				\hline
				$\begin{pmatrix}
					1 & 5 & 7\\
					2 & 6 & 8\\
					3 & 7 & 9
				\end{pmatrix}$ 
				&  
				$\begin{pmatrix}
					&
				\end{pmatrix}$
				&
				\text{Сообщение об ошибке}
				&
				\text{Сообщение об ошибке} \\ 
				\hline
				$\begin{pmatrix}
					1 & 5 & 7\\
				\end{pmatrix}$ 
				&  
				$\begin{pmatrix}
					1 & 2 & 3\\
				\end{pmatrix}$
				&
				\text{Сообщение об ошибке}
				&
				\text{Сообщение об ошибке} \\ 
				\hline
				$\begin{pmatrix}
					1 & 2 & 3\\
					4 & 5 & 6 \\
					7 & 8 & 9 \\
				\end{pmatrix}$ 
				&  
				$\begin{pmatrix}
					1 & 0 & 0\\
					0 & 1 & 0 \\
					0 & 0 & 1 \\
				\end{pmatrix}$
				&
				$\begin{pmatrix}
					1 & 2 & 3\\
					4 & 5 & 6 \\
					7 & 8 & 9 \\
				\end{pmatrix}$ 
				&
				$\begin{pmatrix}
					1 & 2 & 3\\
					4 & 5 & 6 \\
					7 & 8 & 9 \\
				\end{pmatrix}$ \\ 
				\hline
				$\begin{pmatrix}
					1 & 2 & 3 \\
					4 & 5 & 6
				\end{pmatrix}$
				&
				$\begin{pmatrix}
					1 \\
					10 \\
					100
				\end{pmatrix}$
				&
				$\begin{pmatrix}
					321 \\
					654
				\end{pmatrix}$ 
				&
				$\begin{pmatrix}
					321 \\
					654
				\end{pmatrix}$ \\ 
				\hline
				$\begin{pmatrix}
					10
				\end{pmatrix}$
				&
				$\begin{pmatrix}
					35
				\end{pmatrix}$
				&
				$\begin{pmatrix}
					350
				\end{pmatrix}$ 
				&
				$\begin{pmatrix}
					350
				\end{pmatrix}$ \\ 
				\hline
			\end{tabular}
		\end{threeparttable}
	\end{center}
\end{table}

Все тесты пройдены успешно.

\section*{Вывод}

В данном разделе были перечислены использованные технические средства, приведены и протестированы реализации трёх алгоритмов умножения матриц. В результате тестирования корректность реализаций была подтверждена.

\clearpage
