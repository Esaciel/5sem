\ssr{ЗАКЛЮЧЕНИЕ}

Все поставленные задачи, а именно:
\begin{enumerate}[label={\arabic*)}]
	\item описание последовательного алгоритма решения задачи согласно персональному варианту;
	\item разработка параллельной версии алгоритма;
	\item реализация обеих версий алгоритма;
	\item сравнительный анализ зависимостей времени решения задачи от размерности входа для реализации последовательного алгоритма и для реализации модифицированного алгоритма, запущенного с единственным рабочим потоком;
	\item сравнительный анализ зависимостей времени решения задачи от размерности входа для реализации модифицированного алгоритма при $k$ рабочих потоках, $k$ принимает значения $1,2,\cdots,8\cdot q$, где $q$ --- количество логических ядер процессора ЭВМ;
	\item формулировка рекомендаций о выборе $k$ для решения задачи на ЭВМ.
\end{enumerate}

Были выполнены.

Цель лабораторной работы, заключающаяся в разработке и сравнительном анализе последовательного и параллельного алгоритмов, была достигнута.

\clearpage