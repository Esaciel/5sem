\chapter{Исследовательская часть}

\section{Характеристики ЭВМ}

В листинге~\ref{lst:sysinfo} приведены характеристики ЭВМ и системы, использованных для проведения замеров времени выполнения реализаций алгоритмов.

\lstinputlisting[label=lst:sysinfo,caption=Характеристика ЭВМ --- частичный вывод комманд \textit{systeminfo.exe} и \textit{wmic cpu}, captionpos=b]{listings/sysinfo.txt}

\section{Время выполнения алгоритмов}

Время работы каждой реализации считалось как среднее арифметическое из 100 повторений. На вход подавались граф с числом вершин от 10 до 2000, для параллельного алгоритма --- число потоков от $2^1$ до $2^7$.

В таблицах~\ref{tbl:time1}--\ref{tbl:time3} приведены результаты замеров времени выполнения последовательного алгоритма и параллельного алгоритма с различным числом потоков при различном количестве вершин во входном графе. Число потоков равное $0$ соответствует времени выполнения последовательной реализации алгоритма.

\begin{table}[ht]
	\small
	\begin{center}
		\begin{threeparttable}
			\caption{Замер времени выполнения реализаций, часть 1}
			\label{tbl:time1}
			\begin{tabular}{|r|S|S|S|}
				\hline
				& \multicolumn{3}{c|}{\bfseries Число потоков} \\
				\cline{2-4}
				\bfseries \makecell{Размер графа} & \bfseries 0 & \bfseries 1 & \bfseries 2 \\
				\cline{2-4}
				\hline
				10 & 2.30882e-05 & 0.000130843 & 0.000129115 \\
				\hline
				50 & 0.000892585 & 0.00100413 & 0.00128785 \\
				\hline
				100 & 0.00693508 & 0.00756251 & 0.00788914 \\
				\hline
				300 & 0.169208 & 0.181418 & 0.150983 \\
				\hline
				500 & 0.835674 & 0.851258 & 0.679207 \\
				\hline
				1000 & 6.72219 & 6.79104 & 5.50954 \\
				\hline
				2000 & 52.3328 & 55.4725 & 43.5272 \\
				\hline
			\end{tabular}	
		\end{threeparttable}
	\end{center}
\end{table}

\begin{table}[ht]
	\small
	\begin{center}
		\begin{threeparttable}
			\caption{Замер времени выполнения реализаций, часть 2}
			\label{tbl:time2}
			\begin{tabular}{|r|S|S|S|}
				\hline
				& \multicolumn{3}{c|}{\bfseries Число потоков} \\
				\cline{2-4}
				\bfseries \makecell{Размер графа} & \bfseries 4 & \bfseries 8 & \bfseries 16 \\
				\cline{2-4}
				\hline
				10 & 0.000218367 & 0.000593067 & 0.00107533 \\
				\hline
				50 & 0.00137632 & 0.000886386 & 0.00126667 \\
				\hline
				100 & 0.00845587 & 0.00493297 & 0.00411563 \\
				\hline
				300 & 0.143859 & 0.0755773 & 0.0525234 \\
				\hline
				500 & 0.668956 & 0.351325 & 0.220384 \\
				\hline
				1000 & 5.36154 & 2.85611 & 1.91915 \\
				\hline
				2000 & 41.5779 & 23.4642 & 16.3222 \\
				\hline
			\end{tabular}	
		\end{threeparttable}
	\end{center}
\end{table}

\begin{table}[ht]
	\small
	\begin{center}
		\begin{threeparttable}
			\caption{Замер времени выполнения реализаций, часть 3}
			\label{tbl:time3}
			\begin{tabular}{|r|S|S|S|}
				\hline
				& \multicolumn{3}{c|}{\bfseries Число потоков} \\
				\cline{2-4}
				\bfseries \makecell{Размер графа} & \bfseries 32 & \bfseries 64 & \bfseries 128 \\
				\cline{2-4}
				\hline
				10 & 0.00596604 & 0.00688162 & 0.00851111 \\
				\hline
				50 & 0.00281614 & 0.00552692 & 0.014393 \\
				\hline
				100 & 0.0039265 & 0.006561 & 0.0167919 \\
				\hline
				300 & 0.0407727 & 0.0375035 & 0.0375762 \\
				\hline
				500 & 0.162833 & 0.146575 & 0.139394 \\
				\hline
				1000 & 1.41774 & 1.22245 & 1.20176 \\
				\hline
				2000 & 12.0204 & 11.3218 & 8.18124 \\
				\hline
			\end{tabular}	
		\end{threeparttable}
	\end{center}
\end{table}

\textbf{Замечание:} вопреки заданию к лабораторной работе, замеры времени выполнения для графов с числом вершин, равным $3000$, не были произведены. При попытке их проведения, процессор ЭВМ перегревался, что приводило к аварийному завершению работы. Из-за опасности для ЭВМ, а также из-за низкой степени доверия к результатам, полученным в подобных условиях, после 2-х безуспешных попыток было принято решение не проводить замеры на графах размером более $2000$ вершин.

На рисунке~\ref{img:measure} приведён графики времён выполнения реализаций от размера графа.

\begin{figure}[H]
	\centering
	\includegraphics[width=1\textwidth]{images/measure.pdf}
	\caption{Сравнение реализаций последовательного алгоритма и параллельного алгоритма с разным числом потоков}
	\label{img:measure}
\end{figure}

\section*{Вывод}

В данной части были проведены исследования зависимости реального времени выполнения реализаций последовательного алгоритма и параллельного алгоритма с разным числом потоков от числа вершин в графе. По результатам замеров, представленных в таблицах~\ref{tbl:time1}--\ref{tbl:time3} установлено следующее:
\begin{enumerate}[label=---]
	\item при числе вершин в графе $ \leqslant 50$, реализация последовательного алгоритма выполняется быстрее реализации параллельного с любым числом потоков. Это связано с накладными расходами на создание и завершение потоков, а также с дополнительными вычислениями внутри потоков, возникающими из-за особенностей блочного алгоритма;
	\item при числе вершин $ = 100$, быстрее всех выполняется параллельная реализация с $32$-я потоками;
	\item при числе вершин $ = 300$, быстрее всех выполняется параллельная реализация с $64$-я потоками;
	\item при числе вершин $ \geqslant 300$, быстрее всех выполняется параллельная реализация с максимальным числом потоков (равным $128$).
\end{enumerate}

А также:
\begin{enumerate}
	\item при числе вершин равном $100$, реализация последовательного алгоритма выполняется быстрее реализации параллельного с числом потоков в только промежутке $1$--$4$.
	\item при большем числе вершин во входном графе, последовательная реализация всегда выигрывает только у параллельной реализации с одним потоком, т.к. при одном рабочем потоке алгоритм идентичен последовательному, но добавляет накладные расходы на создание потока.
\end{enumerate}

\clearpage
