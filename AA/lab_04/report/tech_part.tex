\chapter{Технологическая часть}

В данном разделе будут перечислены средства реализации, листинги кода
и функциональные тесты.

\section{Средства реализации}

Требования к языку программирования:

\begin{enumerate}[label=---]
	\item статическая типизация;
	\item гарантированное время выполнения.
\end{enumerate}

В качестве языка программирования, соответствующего указанным требованиям, для выполнения данной лабораторной работы был выбран С++~\cite{book:cpp}.

Для замеров реального времени используется \textit{high\_resolution\_clock}~\cite[с.~1016]{book:cpp} библиотеки \textit{chrono}~\cite[с.~1009--1018]{book:cpp}.

Для создания автоматически присоединяемых потоков с возможностью отправки в них запроса остановки используется \textit{jthread}~\cite[с.~244]{book:jthread}.

Для создания мьютекса используется \textit{mutex} библиотеки \textit{mutex}~\cite[с.~1220]{book:cpp}.

Для создания блокировки используются \textit{lock\_guard::lock()} библиотеки \textit{mutex}~\cite[с.~1220]{book:cpp}.

Для создания переменных условий используется \textit{condition\_variable} из одноимённой библиотеки~\cite[c.~1231]{book:cpp}.

\section{Реализация алгоритмов}

В листинге~\ref{lst:sequential} приведена реализация последовательного алгоритма поиска пар вершин графа, расстояние между которыми меньше заданной действительной величины.

В листинге~\ref{lst:parallel} приведена реализация главного потока параллельного алгоритма поиска пар вершин графа, расстояние между которыми меньше заданной действительной величины.

В листинге~\ref{lst:par_worker} приведена реализация рабочего потока параллельного алгоритма поиска пар вершин графа, расстояние между которыми меньше заданной действительной величины --- алгоритма обработки одного блока матрицы минимальных расстояний.

В листинге~\ref{lst:threadpool} приведены реализации методов пула потоков, необходимых для работы параллельного алгоритма.

\lstinputlisting[label=lst:sequential,caption=Реализация последовательного алгоритма, captionpos=b]{listings/sequential.txt}

\lstinputlisting[label=lst:parallel,caption=Реализация главного потока параллельного алгоритма, captionpos=b]{listings/parallel.txt}

\lstinputlisting[label=lst:par_worker,caption=Реализация рабочего потока параллельного алгоритма, captionpos=b]{listings/par_worker.txt}

\lstinputlisting[label=lst:threadpool,caption=Реализации методов пула потоков, captionpos=b]{listings/threadpool.txt}

\section{Функциональные тесты}

Тестирование проведено по методологии чёрного ящика. Функциональные тесты приведены в таблице~\ref{tbl:func_tests}

\begin{table}[ht]
	\small
	\begin{center}
		\begin{threeparttable}
			\caption{Функциональные тесты для реализаций алгоритмов поиска пар вершин на расстоянии меньшем, чем данная на вход действительная величина}
			\label{tbl:func_tests}
			\begin{tabular}{|p{0.05\textwidth}|p{0.14\textwidth}|p{0.14\textwidth}|p{0.10\textwidth}|p{0.08\textwidth}|p{0.14\textwidth}|p{0.14\textwidth}|}
				\hline
				№ &  Описание &  Массив граней &  Веса &  Порог &  Обычный алгоритм &  Параллельный алгоритм \\
				\hline
				1 &  Несвязанный граф &  () &  () &  1 & [ ] & [ ] \\
				\hline
				2 &  Связанный граф без искомых пар вершин &  ((1, 2), (2, 3), (3, 1)) &  (3, 4, 5) &  2 &  [ ] &  [ ] \\
				\hline
				3 &  Связанный граф с имеющимися искомыми парами &  ((1, 2), (2, 3), (3, 4), (1, 3)) &  (2, 1, 3, 4) &  3 &  [[1, 2], [2, 3], [1, 3]] &  [[1, 2], [2, 3], [1, 3]] \\
				\hline
			\end{tabular}
		\end{threeparttable}
	\end{center}
\end{table}

Все тесты пройдены успешно.

\section*{Вывод}

В данном разделе были перечислены использованные технические средства, приведены и протестированы реализации последовательного и параллельного алгоритмов поиска пар вершин на расстоянии меньше заданной величины. В результате тестирования корректность реализаций была подтверждена.

\clearpage
